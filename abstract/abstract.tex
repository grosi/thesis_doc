%%%%%%%%%%%%%%%%%%%%%%%%%%%%%%%%%%%%%%%%%%%%%%%%%%%%%%%%%%%%%%%%%%%%%%%%%%%%%%%
% Titel:   Abstract
% Autor:   zursr1, gross10
% Datum:   28.05.2014
% Version: 1.0.0
%%%%%%%%%%%%%%%%%%%%%%%%%%%%%%%%%%%%%%%%%%%%%%%%%%%%%%%%%%%%%%%%%%%%%%%%%%%%%%%

%:::Change-Log:::
% Versionierung erfolgt auf folgende Gegebenheiten: -1. Stelle Semester
%                                                   -2. Stelle neuer Inhalt
%                                                   -3. Fehlerkorrekturen
%
% 1.0.0       Erstellung der Datei

%%%%%%%%%%%%%%%%%%%%%%%%%%%%%%%%%%%%%%%%%%%%%%%%%%%%%%%%%%%%%%%%%%%%%%%%%%%%%%%
\chapter*{Management Summary}
Nach jeder Verletzung will der Sportler schnellstm�glich mit dem Training weiterfahren. In jedem Fall sollte die Verletzung jedoch bestm�glich geheilt sein, um Folgesch�den zu vermeiden. An der Eidgen�ssischen Hochschule f�r Sport in Magglingen wird bei Kreuzbandrissen ein Test mit einen Physiotherapeuten am verletzen Sportler durchgef�hrt. Dabei absolviert der Sportler diverse �bungen mit unterschiedlichen Belastungen des Knies. Die seitliche Auslenkung des Knies l�sst auf dessen Genesungszustand schliessen. Dieser Test ist durch seine Subjektivit�t nicht die bestm�gliche L�sung. Eine falsche Auswertung hat bei zu fr�hem Trainingseinstieg Folgen auf die Gesundheit des Athleten. Eine zus�tzliche Angabe der absoluten Auslenkung w�re f�r die Therapeuten hilfreich. Ziel der Arbeit ist ein Konzept zu entwickelt damit der Therapeut eine Basis f�r eine begr�ndete Diagnose hat. Dabei soll ein 3D-Beschleunigungssensor verwendet werden.\par
%
Auf dem Gebiet der Biomechanik wird viel Forschung zur Untersuchung eines Bewegungsablaufes mit 3D-Beschleunigungssensoren betrieben. In solchen Publikationen wurden diverse Ans�tze gefunden. Ebenfalls wurde ein Beispielprojekt in LabVIEW entdeckt, in welchem bereits die Sch�tzung der Orientierung des Bewegungssensors im Raum umgesetzt ist. In MATLAB wurde das Konzept basierend auf der Literaturstudie und dem Beispielprojekt entwickelt. In diesem Konzept wird in einer anf�nglichen Kalibrationsphase die Orientierung des Sensors gesch�tzt. Mit dieser Sch�tzung kann die Orientierung durch eine Rotation kompensiert werden. Durch diese Rotation wird die Erdbeschleunigung von der Beschleunigung der Bewegung des Knies getrennt. Auf Basis eines bekannten physikalisch grundlegenden Gesetzes kann durch zweifache Integration der Beschleunigung auf den zur�ckgelegten Weg geschlossen werden. Mehrere Messreihen wurden aufgenommen um das Konzept zu �berpr�fen. Das Konzept wurde darauf teilweise in LabVIEW umgesetzt.\par
%
Das entwickelte Konzept zeichnet die Auslenkung des Knies in einer Ebene senkrecht zur Erdbeschleunigung auf. Die Kompensation der Orientierung des Sensors ist zuverl�ssig. Die Bewegung des Knies in der definierten Ebene kann ebenfalls berechnet werden (siehe Abbildung). Das Konzept ist jedoch nicht f�hig, Auslenkungen zu detektieren, die diese Ebene verlassen. Wird der Sensor nach der Kalibration zus�tzlich rotiert, werden die Beschleunigungsdaten durch die Erdbeschleunigung verf�lscht. Diese Verf�lschung wird mit integriert und f�hrt zu einem Drift. Aus diesem Grund wurde die Entwicklung eines neuen Ansatzes gestartet, bei welchem die Bewegung mittels eines Models in einen dynamischen und einen statischen Fall unterteilt werden kann. Die Bewegung des Knies kann als eine harmonische Funktion approximiert werden. Somit ist auch die Beschleunigung der Bewegung eine harmonische Funktion. Dabei zeigt sich, dass bei dynamischen Bewegungen oberhalb einer kritischen Frequenz der Einfluss der Gravitation auf die Beschleunigung der Bewegung vernachl�ssigt werden kann. Daraus l�sst sich einen einfachen Algorithmus ableiten, was Bestandteil eines sp�teren Projektes sein wird.
