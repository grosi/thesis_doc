%%%%%%%%%%%%%%%%%%%%%%%%%%%%%%%%%%%%%%%%%%%%%%%%%%%%%%%%%%%%%%%%%%%%%%%%%%%%%%%
% Titel:   Bericht - CD
% Autor:   gross10
% Datum:   19.06.2014
% Version: 0.0.1
%%%%%%%%%%%%%%%%%%%%%%%%%%%%%%%%%%%%%%%%%%%%%%%%%%%%%%%%%%%%%%%%%%%%%%%%%%%%%%%
%
%:::Change-Log:::
% Versionierung erfolgt auf folgende Gegebenheiten: -1. Release Versionen
%                                                   -2. Neue Kapitel
%                                                   -3. Fehlerkorrekturen
%
% 0.0.01      Erstellung der Datei
%%%%%%%%%%%%%%%%%%%%%%%%%%%%%%%%%%%%%%%%%%%%%%%%%%%%%%%%%%%%%%%%%%%%%%%%%%%%%%% 
\chapter{CD}\label{ch:cd}
	Aus �kologischen und �konomischen Gr�nden wird ein grosser Teil des Anhangs in Form einer CD-ROM dieser Dokumentation beigelegt. Die Dateistruktur entspricht dabei der Gliederung dieses Kapitels.
    %
    \section{LabVIEW}\label{s:anhang_labview}
    	Alle ben�tigten \gls{g:labview}-Projekte:
    	%
    	\begin{itemize}
	    	\item \textbf{Konzept}: \texttt{concept\_reametric}
	    	\item \textbf{Implementierung}: \texttt{reametric}
    	\end{itemize}
    %
    \section{MATLAB}\label{s:anhang_matlab}
    	Die \gls{g:matlab}-Skripte die w�hrende des Projekts verwendet wurden.
    	\begin{itemize}
	    	\item \texttt{Check\_Rotation\_final.m}
	    	\item \texttt{Check\_Translation\_Rotation\_final.m}
	    	\item \texttt{Dynamisch\_Statisch.m}
	    	\item \texttt{Konzept\_Translation\_final.m}
    	\end{itemize}
    %
    \section{Quellen Internet}\label{s:anhang_quellen_internet}
    	Sicherungen der Internetquellen.
    	\begin{itemize}
	    	\item Anterior Cruciate Ligament Injury
	    	\item How Do I Make an XY Graph Behave as an XY Chart
	    	\item Lossless Communication with Network Streams Components Architecture and Performance
	    	\item MATLAB script
	    	\item NI Linux Real Time unter der Lupe
	    	\item What Is Measurement Automation Explorer MAX
    	\end{itemize}
    %
    \section{Datenblatt NI myRIO}\label{s:anhang_datasheets}
    	
    	
  