%%%%%%%%%%%%%%%%%%%%%%%%%%%%%%%%%%%%%%%%%%%%%%%%%%%%%%%%%%%%%%%%%%%%%%%%%%%%%%%
% Titel:   Bericht - Pflichtenheft
% Autor:   gross10
% Datum:   13.12.2013
% Version: 0.0.1
%%%%%%%%%%%%%%%%%%%%%%%%%%%%%%%%%%%%%%%%%%%%%%%%%%%%%%%%%%%%%%%%%%%%%%%%%%%%%%%
%
%:::Change-Log:::
% Versionierung erfolgt auf folgende Gegebenheiten: -1. Release Versionen
%                                                   -2. Neue Kapitel
%                                                   -3. Fehlerkorrekturen
%
% 0.0.01      Erstellung der Datei
%%%%%%%%%%%%%%%%%%%%%%%%%%%%%%%%%%%%%%%%%%%%%%%%%%%%%%%%%%%%%%%%%%%%%%%%%%%%%%% 
\chapter{Auslesen DAQ-Karte mit MATLAB}\label{ch:daqauslesen}
\gls{g:matlab} bietet zum direkten Auslesen von Daten eine Data Acquisition Toolbox an. In dieser k�nnen angeschlossene \acrshort{ac:daq}-Karten in einer sogenannten Session direkt in \gls{g:matlab} beispielsweise mittels USB ausgelesen werden. Auch k�nnen Daten/Signale, welche in \gls{g:matlab} generiert wurden �ber Ausg�nge der Karte ausgegeben werden.\par 
Zum Auslesen eines Beschleunigungssensors wurde ein kurzes \gls{g:matlab}-Skript geschrieben. Dieses Skript sollte auch zuk�nftigen Arbeiten eine schnelle und einfache Verwendung der Beschleunigungssensoren erm�glichen.\par
\begin{lstlisting}[style=Matlab,caption={DAQ\_Auslesen.m},label={list:daqauslesen}]
%**************************************************************************
%   Smart Rehabilitation Thesis 2014 
%      
%   Skript zum initialisieren und auslesen des ICP-Beschleunigungssensors
%   603C01 von IMI Sensors mit der NI USB-4431 DAQ-Karte
%   
%   Reto Zurschmiede, Juni 2014
%**************************************************************************

clear all
close all
clc

% Laden des verwendeten Devices 
devices = daq.getDevices

% Port einstellen
devices(1)

% Session f�r NI Karte erstellen
s = daq.createSession('ni');

% Analogeingang hinzuf�gen
[ai0, ind0] = s.addAnalogInputChannel(devices.ID,'ai0','Accelerometer');

% Sensor initialisieren
s.Channels(ind0).Sensitivity = 100e-3;  % Sensitivit�t des Sensors [V/g]
ai0.Coupling = 'AC';                    % DC-Bias auskoppeln
ai0.ExcitationSource = 'Internal';      % Speisung des Sensors
ai0.ExcitationCurrent = 2e-3;           % Konstant Strom [A] 

% Abtastfrequenz einstellen [Hz]
s.Rate = 20e3;

% Aufzeichnungsdauer einstellen [s]
s.DurationInSeconds = 5;

% Daten auslesen
[data, time] = s.startForeground;

% Daten darstellen
figure 
plot(time,data)
xlabel('Time [s]')
ylabel('Acceleration [g]')
\end{lstlisting}
%
%Datenaufnahme mit ICP-Beschleunigungssensoren
\section*{Datenaufnahme mit ICP-Beschleunigungssensoren}
Um bei der Konzeptentwicklung Daten zum �berpr�fen des Algorithmus zu haben, erhielten wir von Rolf Vetter \gls{ac:icp}-Beschleunigungssensoren von IMI Sensors. Mit einer \gls{ac:daq}-Karte von \gls{ac:ni}, der \gls{ac:ni} USB-4431, wurden die Beschleunigungssensoren direkt mittels MATLAB ausgelesen. Somit sind die Daten direkt in MATLAB zur Weiterverarbeitung abgelegt. Es hat sich jedoch schnell herausgestellt, dass die Sensoren f�r unsere Anwendung nicht geeignet sind. Der Frequenzbereich des Sensors ist im Datenblatt in \vref{ch:indusriesensor} ersichtlich. Dort zu sehen ist, dass der Sensor einen Frequenzbereich von \unit[0.5]{Hz} bis \unit[10]{kHz} hat. Somit wird die f�r uns notwendige Gravitationsbeschleunigung, die einem DC-Anteil im Signal entspricht, nicht aufgenommen. Diese Sensoren k�nnen nur f�r dynamische Anwendungen verwendet werden wie Beispielsweise einer Ersch�tterungsmessung.

