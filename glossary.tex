%%%%%%%%%%%%%%%%%%%%%%%%%%%%%%%%%%%%%%%%%%%%%%%%%%%%%%%%%%%%%%%%%%%%%%%%%%%%%%%
% Titel:   Bericht - Glossar
% Autor: Simon Grossenbacher  
% Datum:   05.10.2013
% Version: 1.0.0
%%%%%%%%%%%%%%%%%%%%%%%%%%%%%%%%%%%%%%%%%%%%%%%%%%%%%%%%%%%%%%%%%%%%%%%%%%%%%%%
%
%:::Change-Log:::
% Versionierung erfolgt auf folgende Gegebenheiten: -1. Release Versionen
%                                                   -2. Neue Kapitel
%                                                   -3. Fehlerkorrekturen
%
% 0.0.0       Erstellung der Datei
%
%:::Hinweis:::
% Indexerstellung: makeindex -s report.ist report.idx
%   Umlaute m�ssen separat behandelt werden!
%%%%%%%%%%%%%%%%%%%%%%%%%%%%%%%%%%%%%%%%%%%%%%%%%%%%%%%%%%%%%%%%%%%%%%%%%%%%%%%

\newglossaryentry{g:beispiel}{name=Beispiel, description={Ein Beispiel}}
%
% Triangulation
\newglossaryentry{g:triangulation}{name=Triangulation, description= {Positionsbestimmung im Raum mittels Richtungs- und Abstandsmessung. Berechnung der Position erfolgt anschliessend mit geometrischen Funktionen.}}
%
% Magnetometer
\newglossaryentry{g:magnetometer}{name=Magnetometer, description={Sensor zum Messen von magnetischen Flussdichten}}
%
% Transversalebene
\newglossaryentry{g:transversalebene}{name=Transversalebene, description={Ebene in der Medizin. Erstreckt sich von links nach rechts und von vorne nach hinten. Sie steht senkrecht auf der L�ngsachse des Mesnchlichen K�rpers. \cite{lit:transversal}}}
%
% LabVIWE
\newglossaryentry{g:labview}{name=LabVIEW, description={System Design Software von der Firma \acrfull{ac:ni}}}
%
% myRIO
\newglossaryentry{g:myrio}{name=NI myRIO, description={Developmentboard der Firma \acrfull{ac:ni} entwickelt f�r den g�nstigen Einstieg in \gls{g:labview}}}
%
% MATLAB
\newglossaryentry{g:matlab}{name=MATLAB, description={Software von \href{http://www.mathworks.ch/}{The MathWorks} speziel ausgelegt zum L�sen mathematischer Problemstellungen}}