%%%%%%%%%%%%%%%%%%%%%%%%%%%%%%%%%%%%%%%%%%%%%%%%%%%%%%%%%%%%%%%%%%%%%%%%%%%%%%%
% Titel:   Bericht - Glossar
% Autor: Simon Grossenbacher  
% Datum:   05.10.2013
% Version: 1.0.0
%%%%%%%%%%%%%%%%%%%%%%%%%%%%%%%%%%%%%%%%%%%%%%%%%%%%%%%%%%%%%%%%%%%%%%%%%%%%%%%
%
%:::Change-Log:::
% Versionierung erfolgt auf folgende Gegebenheiten: -1. Release Versionen
%                                                   -2. Neue Kapitel
%                                                   -3. Fehlerkorrekturen
%
% 0.0.0       Erstellung der Datei
%
%:::Hinweis:::
% Indexerstellung: makeindex -s report.ist report.idx
%   Umlaute m�ssen separat behandelt werden!
%%%%%%%%%%%%%%%%%%%%%%%%%%%%%%%%%%%%%%%%%%%%%%%%%%%%%%%%%%%%%%%%%%%%%%%%%%%%%%%

\newglossaryentry{g:beispiel}{name=Beispiel, description={Ein Beispiel}}
\newglossaryentry{g:triangulation}{name=Triangulation, description= {Positionsbestimmung im Raum mittels Richtungs- und Abstandsmessung. Berechnung der Position erfolgt anschliessend mit geometrischen Funktionen.}}
\newglossaryentry{g:magnetometer}{name=Magnetometer, description={Sensor zum Messen von magnetischen Flussdichten}}
\newglossaryentry{g:transversalebene}{name=Transversalebene, description={Ebene in der Medizin. Erstreckt sich von links nach rechts und von vorne nach hinten. Sie steht senkrecht auf der L�ngsachse des Mesnchlichen K�rpers. \cite{lit:transversal}}}




