%%%%%%%%%%%%%%%%%%%%%%%%%%%%%%%%%%%%%%%%%%%%%%%%%%%%%%%%%%%%%%%%%%%%%%%%%%%%%%%
% Titel:   Bericht - Glossar
% Autor: Simon Grossenbacher  
% Datum:   05.10.2013
% Version: 1.0.0
%%%%%%%%%%%%%%%%%%%%%%%%%%%%%%%%%%%%%%%%%%%%%%%%%%%%%%%%%%%%%%%%%%%%%%%%%%%%%%%
%
%:::Change-Log:::
% Versionierung erfolgt auf folgende Gegebenheiten: -1. Release Versionen
%                                                   -2. Neue Kapitel
%                                                   -3. Fehlerkorrekturen
%
% 0.0.0       Erstellung der Datei
%
%:::Hinweis:::
% Indexerstellung: makeindex -s report.ist report.idx
%   Umlaute m�ssen separat behandelt werden!
%%%%%%%%%%%%%%%%%%%%%%%%%%%%%%%%%%%%%%%%%%%%%%%%%%%%%%%%%%%%%%%%%%%%%%%%%%%%%%%

\newglossaryentry{g:beispiel}{name=Beispiel, description={Ein Beispiel}}
%
% Triangulation
\newglossaryentry{g:triangulation}{name=Triangulation, description= {Positionsbestimmung im Raum mittels Richtungs- und Abstandsmessung. Berechnung der Position erfolgt anschliessend mit geometrischen Funktionen.}}
%
% Magnetometer
\newglossaryentry{g:magnetometer}{name=Magnetometer, description={Sensor zum Messen von magnetischen Flussdichten}}
%
% Sagitalebene
\newglossaryentry{g:sagitalebene}{name=Sagitalebene, description={Ebene in der Medizin. Erstreckt sich von vorne nach hinten und von oben nach unten. Sie steht senkrecht auf der Transversalebene. \cite{lit:transversal}}}
%
% Transversalebene
\newglossaryentry{g:transversalebene}{name=Transversalebene, description={Ebene in der Medizin. Erstreckt sich von links nach rechts und von vorne nach hinten. Sie steht senkrecht auf der L�ngsachse des menschlichen K�rpers. \cite{lit:transversal}}}
%
% sagital
\newglossaryentry{g:sagital}{name=sagital,description={Bewegung in der Sagitalebene, vorne nach hinten umgekehrt}}
%
% transversal
\newglossaryentry{g:transversal}{name=transversal,description={Bewegung in der Transversalebene, rechts nach links umgekehrt}}

%
% LabVIEW
\newglossaryentry{g:labview}{name=LabVIEW, description={System Design Software von der Firma \acrfull{ac:ni}}}
%
% myRIO
\newglossaryentry{g:myrio}{name=NI myRIO, description={Developmentboard der Firma \acrfull{ac:ni} entwickelt f�r den g�nstigen Einstieg in \gls{g:labview}}}
%
% ssh
\newglossaryentry{g:ssh}{name=SSH, description={Steht f�r Secure Shell und stellt eine g�ngige M�glichkeit dar eine sicher Verbindung zu einem UNIX-System herzustellen}}
%
% webdav
\newglossaryentry{g:webdav}{name=WebDAV, description={Steht f�r Web-based Distributed Authoring and Versioning und stellt ein offener Standard f�r den Datentransfer im Internet dar}}
%
% MATLAB
\newglossaryentry{g:matlab}{name=MATLAB, description={Software von \href{http://www.mathworks.ch/}{The MathWorks} speziel ausgelegt zum L�sen mathematischer Problemstellungen}}
%
% station�res Signal
\newglossaryentry{g:stationaer}{name=station�r, description={Ein Signal wird als station�r bezeichnet, wenn seine Signalkenngr�ssen nicht abh�ngig von der Zeit sind}}
%
% Gradientenverfahren
\newglossaryentry{g:gradientenvefahren}{name=Gradientenvefahren, description={Verfahren bei Optimierungen. Dabei wird der  angen�herter Wert stets in die Richtung des negativen Gradienten korrigiert (Methode des steilsten Abstieges)}}
%
% Body frame
\newglossaryentry{g:bodyframe}{name=Body-Frame, description={Das objektbezogene Koordinatensystem des myRIO}}
%
% world Frame
\newglossaryentry{g:worldframe}{name=World-Frame, description={Das festes Referenzsystem der Erde}}
%
% LTI
\newglossaryentry{g:lti}{name=LTI,description={linear time-invariant, System das sowohl \gls{g:linear} sowie \gls{g:zeit_invariant} ist}}
%
% linear
\newglossaryentry{g:linear}{name=linear,description={Summe von gewichteten Eingangssignalen wird zu Summe von gleich gewichteten Ausgangssignalen verarbeitet}}
%
% zeit_invariant
\newglossaryentry{g:zeit_invariant}{name=zeit-invariant,description={System ist zeitunabh�ngig, ein um $\tau$ verz�gertes Signal am Eingang ist auch am Ausgang um $\tau$ verz�gert}}
%
% Gyroskop
\newglossaryentry{g:gyroskop}{name=Gyroskop,description={Inertialmessger�t zum Messen der Winkelgeschwindigkeit um ein Drehachse. Durch Integration der Winkelgeschwindigkeit kann auf den Drehwinkel geschlossen werden}}