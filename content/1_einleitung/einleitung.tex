%%%%%%%%%%%%%%%%%%%%%%%%%%%%%%%%%%%%%%%%%%%%%%%%%%%%%%%%%%%%%%%%%%%%%%%%%%%%%%%
% Titel:   Einleitung
% Autor:   zursr1, gross10
% Datum:   28.05.2014
% Version: 0.0.1
%%%%%%%%%%%%%%%%%%%%%%%%%%%%%%%%%%%%%%%%%%%%%%%%%%%%%%%%%%%%%%%%%%%%%%%%%%%%%%%
%
%:::Change-Log:::
% Versionierung erfolgt auf folgende Gegebenheiten: -1. Release Versionen
%                                                   -2. Neue Kapitel
%                                                   -3. Fehlerkorrekturen
%
% 0.0.0       Erstellung der Datei
%%%%%%%%%%%%%%%%%%%%%%%%%%%%%%%%%%%%%%%%%%%%%%%%%%%%%%%%%%%%%%%%%%%%%%%%%%%%%%%
\chapter{Einleitung}\label{ch:einleitung}
Bewegungensanalysen waren fr�her nur mit optischen Verfahren in speziellen Einrichtungen m�glich. EIne Methode ist, dass ein opisches tracken von Markern. Die Marker sind dabei auf dem Testobjekt angebracht. Die Marker sind an f�r die Rekonstruktion des Bewegungsablaufes relevanten Stellen angebracht. Beispielsweise zum simplen Tracken eines L�ufers (Beine) an H�fte, Knie und Kn�chel. Die Marker sind somit an den beweglichen Gelenken angebracht. Der Bewegungsablauf wird von mehreren Kameras aus verschiedenen Perspektiven gefilmt. Mit Bildverarbeitung und Triangulation kann anschliessend auf die Position des Markers im Raum geschlossen werden. Untersuchungen und Messugen von Bewegungsabl�ufen auf diese Art sind nur in speziell daf�r ausgelegten Laboren durchf�rbar. Durch diese Labore sind die Messungen zum einen ortgebunden und zum andern durch die aufw�ndigen und teuren Kamerasystemen kostspielig. Eine L�sung mit g�nstigen und komfortablen Analyseger�ten ist gefragt womit auch Messungen im t�glichen Umfeld durf�hrbar sind.

Dank der MEMS-Technologie \cite{g:MEMS} ist in den letzten Jahren eine zunehmenden Miniaturisierung von Bewegunssensoren wie Beschleunigunssenoren oder Gyroscopen erfolgt. Eine Bewegung kann somit auch mithilfe eines Beschleunigungssensor gemessen werden. Dieser misst jedoch nicht direkt die Position wie in den beschriebenen Kamera-Systemen, sondern die Beschleunigung. Um die Bewegung im Raum zu messen ben�tigt man die Startposition und Orientierung des Sensors sowie die Beschleunigung und Richtung der Bewegung von diesem Punkt aus. Mit anschliessender doppelter Integration der Beschleunigung kann die neue Position berechnet werden. Somit ist die Realisierung von kompakten und f�r den Tr�ger komfortablen Ger�ten m�glich.

Ein Problem bei diesem Verfahren mit den Beschleunigungssensoren, ist die Genuaigkeit der Messung. Bereits ein kleiner Fehler in der Messung der Beschleunigung resultiert zu einem grossen Fehler in der Endposition. Hinzu kommt, dass der Sensor einen Drift aufweist. Somit sind nur Messungen mit anschliessender Berechnung der Position von kurzer Dauer (enigen Sekunden) zu realisieren. 		

\section{Zweck, Ziel, Intension und Notwendigkeit der Untersuchung}
Der Widereinstieg in das Training f�r einen Sportler mit Kreuzbandriss ist schwer zu bestimmen. Zum einen will der Sportler schnellstm�glich mit dem Training weiterfahren um den Trainingsr�ckstand aufzuholen und den get�tigen Aufbau nicht weiter zu verlieren. Zum andern sollte aber die Verletzung bestm�glich geheilt sein um Folgesch�den und weitere Trainingsr�ckschl�ge zu vermeiden. 

An der Eidgen�ssische Hochschule f�r Sport in Magglingen wird dazu ein Test geleitet durch einen Physiotherapeuten am verletzen Sportler durchgef�hrt. Der Test dienen zum Ermitteln des Genesungszustandes des Sportlers. Bei diesem Test werden an der EHSM mehreren �bungen mit unterschiedlichen Belastungen des Knies durchgef�hrt. Ine Kamera filmt den Sportler zus�tzlich w�hrend den �bungen. Die Auslenkung des Knies in der Frontalebene des K�rpers l�sst auf den Genesungszustand nach einem Kreuzbandriss schliessen. Auch ist ersichlich ob der Sportler eine Schonhaltung einnimmt. Mithilfe der Aufnahme wertet der Therapeut diese Auslenkung aus und bestimmt die weiteren Trainingsm�glichkeiten.

Diser Test mit anschliessender Auswertung der Kameraaufnahmen ist durch seine Subjektivit�t nicht die bestm�gliche L�sung. Die Auslenkung des Knies ist schwer zu bestimmen und kann nur gesch�tz werden. Eine falsche Auswertung hat bei zu fr�hem Trainingseinstieg folgen auf die Gesundheit des Athleten, bei zu sp�tem Wiedereinstieg geht dem Sportler wichtige Zeit verloren. Den Therapeuten w�rde eine zus�tzliche Angabe der absolute Auslenkung helfen, um eine bessere Entscheidung treffen zu k�nnen. Diese Auslenkung kann beispielsweise mit einem Beschleunigungssensor gemessen werden.

\section{Stand der Wissenschaft}

\section{Ablauf und Gliederung des Projekts}