%%%%%%%%%%%%%%%%%%%%%%%%%%%%%%%%%%%%%%%%%%%%%%%%%%%%%%%%%%%%%%%%%%%%%%%%%%%%%%%
% Titel:   Einleitung
% Autor:   zursr1, gross10
% Datum:   28.05.2014
% Version: 0.0.1
%%%%%%%%%%%%%%%%%%%%%%%%%%%%%%%%%%%%%%%%%%%%%%%%%%%%%%%%%%%%%%%%%%%%%%%%%%%%%%%
%
%:::Change-Log:::
% Versionierung erfolgt auf folgende Gegebenheiten: -1. Release Versionen
%                                                   -2. Neue Kapitel
%                                                   -3. Fehlerkorrekturen
%
% 0.0.0       Erstellung der Datei
%%%%%%%%%%%%%%%%%%%%%%%%%%%%%%%%%%%%%%%%%%%%%%%%%%%%%%%%%%%%%%%%%%%%%%%%%%%%%%%
\chapter{Einleitung}\label{ch:einleitung}
	Mit der Einf�hrung beginnt die arabische Seitennummerierung. Eine gute Einleitung sollte folgende Punkte er�rtern:
	\begin{itemize}
		\item Einf�hrung in das Problemfeld
		\item Zweck, Ziel, Intension und Notwendigkeit der Untersuchung
		\item Stand der Wissenschaft
		\item Ablauf und Gliederung des Projekts
	\end{itemize}