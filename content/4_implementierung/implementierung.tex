%%%%%%%%%%%%%%%%%%%%%%%%%%%%%%%%%%%%%%%%%%%%%%%%%%%%%%%%%%%%%%%%%%%%%%%%%%%%%%%
% Titel:   Implementierung
% Autor:   zursr1, gross10
% Datum:   28.05.2014
% Version: 0.0.1
%%%%%%%%%%%%%%%%%%%%%%%%%%%%%%%%%%%%%%%%%%%%%%%%%%%%%%%%%%%%%%%%%%%%%%%%%%%%%%%
%
%:::Change-Log:::
% Versionierung erfolgt auf folgende Gegebenheiten: -1. Release Versionen
%                                                   -2. Neue Kapitel
%                                                   -3. Fehlerkorrekturen
%
% 0.0.0       Erstellung der Datei
%%%%%%%%%%%%%%%%%%%%%%%%%%%%%%%%%%%%%%%%%%%%%%%%%%%%%%%%%%%%%%%%%%%%%%%%%%%%%%%
\chapter{Implementierung}\label{ch:implementierung}\todo{myRIO, FPGA erw�hnen / erl�utern}
%
%
% Analyse Portierung MATLAB 
\section{Analyse Portierung MATLAB}\label{s:analyse_portierung_matlab}\todo{Keine Umsetzung, nur Design und VIs!}
	F�r die Implementierung des im vorgehrigen Kapitels \vref{ch:konzept} konzeptionierten Algorithmus muss das erstellte \gls{g:matlab}-Skript analysiert werden. Es gilt \gls{ac:vi}s zu finden, die m�glichst die gleiche Funktion haben wie die verwendeten Befehle und Toolboxen in \gls{g:matlab}. Dabei wird auf das Verwendenen von \gls{g:matlab}-Skripten innerhalb von \gls{g:labview} verzichtet um die Eigenst�ndigkeit der Implementierung zu gew�hrleisten\footnote{Mit der \gls{g:matlab}-Node ist es m�glich \gls{g:matlab}-Code auzuf�hren. Jedoch werden nicht alle Funktionen unterst�tzt und f�r das Ausf�hren muss lokal \gls{g:matlab} installiert sein \cite{lit:labview_matlab}}. Weiter soll eine Struktur gefunden werden, die den Ablauf des \gls{g:matlab}-Skripts wiederspiegelt.
	%
	\subsection{Struktur}\todo{Zustandsmaschine}
	\subsection{Tiefpass-Filterung}\todo{Implementierung der VIs beschreiben}
	\subsection{Sch�tzung der Orientierung} \todo{Implementierung der VIs beschreiben}
	\subsection{Rotation} \todo{Implementierung der VIs beschreiben}
	\subsection{Bewegungsdetektion} \todo{Implementierung der VIs beschreiben}
	\subsection{Integration} \todo{Implementierung der VIs beschreiben}
%
%
% Demonstrator
\section{Demonstrator}\label{s:demonstrator}
	Der Demonstrator stellt die Umsetzung des zuvor analysierten \gls{g:matlab}-Skripts dar. Er dient haupts�chlich zur Demonstration des entwickelten Algorithmus, soll aber auch dessen Funktionalit�t auf einer embedded Hardware zeigen. Die Implementierung basiert auf dem Beispielprojekt \textsf{Posture Estimation} von \gls{ac:ni}, das der Installation von \gls{g:labview} beiliegt. Die Daten der folgend beschriebenen Umsetzung (Projekt \texttt{reametric}) sind im Anhang \vref{s:anhang_labview} zu finden.\par 
	%
	%
	\subsection{Target}
		\subsubsection*{Zustandswechsel}
			\image{content/4_implementierung/image/button}{scale=.7}{htbp}[Zustandswechsel][img:state_switch]
		%
		\subsubsection*{Initialierungs-Zustand}
			%
			\image{content/4_implementierung/image/init_phase}{scale=.7}{htbp}[Initialiserungs-Zustand][img:init_phase]
			%
			\begin{enumerate}
				\item Filterung \texttt{Filter.vi}
				%
				\item Rotationsmatrix \texttt{Mrot.vi}
					%
					\image{content/4_implementierung/image/mrot}{scale=.7}{htbp}[Rotationsmatrix][img:mrot]
				%
				\item Rotation \texttt{Rot.vi}
					%
					\image{content/4_implementierung/image/rot}{scale=.7}{htbp}[Rotation][img:rot]
				%
				\item Standardabweichung
			\end{enumerate}
		%
		%
		\subsubsection*{Signalverarbeitungs-Zustand}
			%
			\image{content/4_implementierung/image/progress_phase}{scale=.7}{htbp}[Signalverarbeitungs-Zustand][img:sig_phase]
			%
			\begin{enumerate}
				\item Filterung \texttt{Filter.vi}
				%
				\item Rotation \texttt{Rot.vi}
				%
				\item Integration \texttt{Inta.vi}
					%
					\image{content/4_implementierung/image/inta}{scale=.7}{htbp}[Integration][img:inta]
				%
				\item Netwerk-Stream
			\end{enumerate}
		%
		%
		\subsubsection*{Frontpanel}
			%
			\image{content/4_implementierung/image/front_myrio}{scale=1}{htbp}[Frontpanel \gls{g:myrio}][img:front_myrio]
		
	%
	%
	\subsection{Host}
		\subsubsection*{Visualisierung}
			%
			\image{content/4_implementierung/image/net_end_host}{scale=.7}{htbp}[Daten Empfangen][img:rx_data]
			%
			\begin{enumerate}
				\item Daten Extraktion
				\item Translation
				\item x-y-Graph
					%
					\image{content/4_implementierung/image/graph_chart_host}{scale=.7}{htbp}[Graph zu Chart][img:graph]
			\end{enumerate}
		%
		%
		\subsubsection*{Frontpanel}
			%
			\image{content/4_implementierung/image/front_host}{scale=.7}{htbp}[Frontpanel Host][img:front_host]
%
%
% Bedienung
\section{Bedienung}
%
\section{Validierung}