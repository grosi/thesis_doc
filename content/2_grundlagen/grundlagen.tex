%%%%%%%%%%%%%%%%%%%%%%%%%%%%%%%%%%%%%%%%%%%%%%%%%%%%%%%%%%%%%%%%%%%%%%%%%%%%%%%
% Titel:   Grundlagen
% Autor:   zursr1, gross10
% Datum:   28.05.2014
% Version: 0.0.1
%%%%%%%%%%%%%%%%%%%%%%%%%%%%%%%%%%%%%%%%%%%%%%%%%%%%%%%%%%%%%%%%%%%%%%%%%%%%%%%
%
%:::Change-Log:::
% Versionierung erfolgt auf folgende Gegebenheiten: -1. Release Versionen
%                                                   -2. Neue Kapitel
%                                                   -3. Fehlerkorrekturen
%
% 0.0.0       Erstellung der Datei
%%%%%%%%%%%%%%%%%%%%%%%%%%%%%%%%%%%%%%%%%%%%%%%%%%%%%%%%%%%%%%%%%%%%%%%%%%%%%%%
\chapter{Grundlagen}\label{ch:grundlagen}
\section{Anatomie des Knies}
\section{LabVIEW auf NI myRIO}
	\todo{Basisprojekt erkl�ren: gross10}
\section{\acrshort{ac:daq} in MATLAB}
MATLAB bietet zum direkten Auslesen von Daten eine Data Acquisition Toolbox an. In dieser k�nnen angeschlossene \acrshort{ac:daq}-Karten in einer sogenannten Session direkt in MATLAB Beispielsweise mittels USB ausgelesen werden. Auch k�nnen Daten/Signale, welche in MATLAB generiert wurden �ber Ausg�nge der Karte ausgegeben werden.
Zum Auslesen eines Beschleunigungssensors wurde ein kurzes MATLAB-Skript geschrieben (ersichtlich in \vref{ch:daqauslesen}). Dieses Skript sollte auch zuk�nftigen Arbeiten eine schnelle und einfache Verwendung der Beschleunigungssensoren erm�glichen.
\section{Rotationsmatrix}
In der Linearen Algebra bietet eine Rotationsmatrix die M�glichkeit, einen Punkt im Raum um einen bestimmten Winkel zu drehen. In unserem Fall ist eine Rotation eines Punktes um eine Gerade g im Raum n�tig. Die Gerade hat die Richtung $\overrightarrow{n}=$ und verl�uft durch den Ursprung. Ihre Parameterdarstellung ist somit: $\overrightarrow{r}= \lambda*\overrightarrow{n}$  \cite{lit:linalg2}
\section{nummerische Integration}
\subsection{Trapezmethode}
\subsection{Filter}