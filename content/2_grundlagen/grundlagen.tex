%%%%%%%%%%%%%%%%%%%%%%%%%%%%%%%%%%%%%%%%%%%%%%%%%%%%%%%%%%%%%%%%%%%%%%%%%%%%%%%
% Titel:   Grundlagen
% Autor:   zursr1, gross10
% Datum:   28.05.2014
% Version: 0.0.1
%%%%%%%%%%%%%%%%%%%%%%%%%%%%%%%%%%%%%%%%%%%%%%%%%%%%%%%%%%%%%%%%%%%%%%%%%%%%%%%
%
%:::Change-Log:::
% Versionierung erfolgt auf folgende Gegebenheiten: -1. Release Versionen
%                                                   -2. Neue Kapitel
%                                                   -3. Fehlerkorrekturen
%
% 0.0.0       Erstellung der Datei
%%%%%%%%%%%%%%%%%%%%%%%%%%%%%%%%%%%%%%%%%%%%%%%%%%%%%%%%%%%%%%%%%%%%%%%%%%%%%%%
\chapter{Grundlagen}\label{ch:grundlagen}
\section{Anatomie des Knies}
\section{LabVIEW auf NI myRIO}
	\todo{Basisprojekt erkl�ren: gross10}
\section{\acrshort{ac:daq} in MATLAB}
MATLAB bietet zum direkten Auslesen von Daten eine Data Acquisition Toolbox an. In dieser k�nnen angeschlossene \acrshort{ac:daq}-Karten in einer sogenannten Session direkt in MATLAB Beispielsweise mittels USB ausgelesen werden. Auch k�nnen Daten/Signale, welche in MATLAB generiert wurden �ber Ausg�nge der Karte ausgegeben werden.
Zum Auslesen eines Beschleunigungssensors wurde ein kurzes MATLAB-Skript geschrieben (ersichtlich in \vref{ch:daqauslesen}). Dieses Skript sollte auch zuk�nftigen Arbeiten eine schnelle und einfache Verwendung der Beschleunigungssensoren erm�glichen.
\section{Rotationsmatrix}
In der Linearen Algebra bietet eine Rotationsmatrix die M�glichkeit, einen Punkt im Raum um einen bestimmten Winkel $\phi$ zu drehen. In unserem Fall ist eine Rotation eines Punktes um eine Gerade im Raum n�tig. Dies wurde in der Linearen Algebra mit Herrn Jenni im zweiten Semseter behandelt \cite{lit:linalg2}.\par
Die Gerade g hat die Richtung $\overrightarrow{n}=\left(a\;b\;c\right)^{T}$ und verl�uft durch den Ursprung. Ihre Parameterdarstellung ist somit: $\overrightarrow{r}= \lambda*\overrightarrow{n}$ mit $\lambda\;\epsilon\;\mathbb{R}$. Der Punkt P mit den Koordinaten $\left(x\mid y\mid z\right)$ soll um den Winkel $\phi$ im Gegenuhrzeigersinn um die Gerade g in den Punkt P' mit Koordinaten $\left(x'\mid y'\mid z'\right)$ gedreht werden. Der Berechnungsschritt l�sst sich somit wie in \vref{eq:rot} zusammenfassen:
\formula{
            \left(\begin{array}{c}x'\\ y'\\z'\end{array}\right)=R_{g;O;\overrightarrow{n};\phi}*\left(\begin{array}{c}x\\ y\\z\end{array}\right)
        }{
            R_{g;O;\overrightarrow{n};\phi} & Rotationsmatrix um Winkel $\phi$ um die Gerade g}[eq:rot]
            
Die Rotationsmatrix wird dabei wie in \vref{eq:rotmax} aufgestellt.
\formula{
            R_{g;O;\overrightarrow{n};\phi} = \cos(\phi)*\begin{bmatrix}1 & 0 & 0 \\0 & 1 & 0 \\ 0 & 0 & 1 \end{bmatrix}+\frac{(1-\cos(\phi))}{a^{2}+b^{2}+c^{2}}*\begin{bmatrix}a^{2} & ab & ac \\ ab & b^{2} & bc \\ ac & bc & c^{2} \end{bmatrix}\\+\frac{\sin(\phi)}{\sqrt{a^{2}+b^{2}+c^{2}}}*\begin{bmatrix}0 & -c & b \\ c & 0 & -a \\ -b & a & 0 \end{bmatrix}
        }{ 
           }[eq:rotmax]
          
\section{nummerische Integration}
\subsection{Trapezmethode}
\subsection{Filter}