%%%%%%%%%%%%%%%%%%%%%%%%%%%%%%%%%%%%%%%%%%%%%%%%%%%%%%%%%%%%%%%%%%%%%%%%%%%%%%%
% Titel:   Grundlagen
% Autor:   zursr1, gross10
% Datum:   28.05.2014
% Version: 0.0.1
%%%%%%%%%%%%%%%%%%%%%%%%%%%%%%%%%%%%%%%%%%%%%%%%%%%%%%%%%%%%%%%%%%%%%%%%%%%%%%%
%
%:::Change-Log:::
% Versionierung erfolgt auf folgende Gegebenheiten: -1. Release Versionen
%                                                   -2. Neue Kapitel
%                                                   -3. Fehlerkorrekturen
%
% 0.0.0       Erstellung der Datei
%%%%%%%%%%%%%%%%%%%%%%%%%%%%%%%%%%%%%%%%%%%%%%%%%%%%%%%%%%%%%%%%%%%%%%%%%%%%%%%
\chapter{Grundlagen}\label{ch:grundlagen}
%
%
%Anatomie des Knies
\section{Anatomie des Knies}
%
%
%LabVIEW auf NI myRIO
\section{LabVIEW auf NI myRIO}
	In der Konzept- und Implementationsphase\footnote{f�r detaillierte Erl�uterungen zu den Projektphasen sei auf den Abschnitt \vref{ss:ablauf_gliederung_projekt} verwiesen} wird das Werkzeug \gls{g:labview} der Firma \gls{ac:ni} verwendet. \gls{g:labview} ist die Abk�rzung f�r \textsf{Laboratory Virtual Instrument Engineering Workbench} und stellt eine Entwicklungsumgebung und eine grafische Programmiersprache dar. Es wurde in den 70er Jahren von den Mitgr�nder von \gls{ac:ni}, Jim Truchard und Jeff Kodosky entworfen und wird stetig weiter entwickelt. Im Gegensatz zu den g�ngigen Hochsprachen wie C/C++ sind in \gls{g:labview} der Datenfluss im Zentrum und nicht Kontrollstrukturen, soll heissen das Datenflussprinzip wird verfolgt \cite[19-22]{lit:labview}.
	%
	\subsection{Aufbau und Struktur}
		Ein \gls{g:labview}-Projekt besteht aus mindestens einem \gls{ac:vi}, indem sich Funktionsbl�cke oder weitere \gls{ac:vi}s befinden k�nnen. Eine \gls{ac:vi} stellt sich zusammen aus einem Frontpanel und einem Blockdiagramm zusammen. Das Frontpanel stellt dabei die Benutzeroberfl�che dar, dagegen das Blockdiagramm die eigentliche Funktionalit�t beinhaltet.\par
		%
		Ein wichtiger Aspekt im Aufbau eines \gls{g:labview}-Projekts ist die Unterscheidung zwischen dem Host und dem Target. Der Host bezeichnet dabei der Computer mit installiertem \gls{g:labview} und das Target ein zu \gls{g:labview} kompatible embedded Hardware, z.B. eine Datenerfassungskarte. Daten die auf dem Target erfasst werden, befinden sich trotz bestehender Kommunikation zum Host nicht automatisch auf demselben. Sie werden stattdessen auf dem internen Speicher des Targets gespeichert und m�ssen anschliessen aus diesem Speicher extrahiert werden\footnote{dieses Fallbeispiel bezieht sich auf die Hardware \gls{g:myrio}}. Ebenso wird zwischen \gls{ac:vi}s auf dem Host und Target unterschieden. Sie m�ssen f�r beide Systeme separat gestartet werden. Die Aufteilung ist im Projektdialog durch eine unterschiedliche Dateibaumtiefe visualisiert, wie der Abbildung \vref{img:labview_gliederung} zu entnehmen ist.
		%
		\image{content/2_grundlagen/image/labview_gliederung}{scale=0.7}{htbp}[\gls{g:labview}-Gliederung][img:labview_gliederung]
	%
	\subsection{Hardware}\label{ss:myRIO_hw}
		F�r die Datenerfassung wird das \gls{g:myrio} eingesetzt. Es handelt sich dabei um ein g�nstiges Developmentboard f�r den raschen Einstieg in \gls{g:labview}. Es bietet folgende f�r das Projekt notwendige Spezifikationen\footnote{n�here Angaben sind im Datenblatt im Anhang \vref{ch:datasheet_myrio} zu finden}:
		\begin{itemize}
			\item Prozessor/FPGA Xilinx Z-7010 (Cortex A9), 2 Kerne � \unit[667]{MHz}
			\item \unit[256]{MB} Flash, \unit[512]{MB} RAM
			\item USB 2.0 Hi-Speed
			\item Accelerometer, 3 Achsen, $\pm$\unit[8]{g}, \unit[12]{bit} Aufl�sung
		\end{itemize}
		%
		Die gesamte Hardware, inklusive Beschleunigungssensor ist in einem kompakten Geh�use verbaut, wie die Abbildung \vref{img:myrio} zeigt. Die Befestigung kann auf einfache weise mit Hilfe von drei Schrauben erfolgen.
		%
		\image{content/2_grundlagen/image/myrio}{scale=0.2}{htbp}[\gls{g:myrio} \cite{img:myrio}][\gls{g:myrio}][img:myrio]
		%
		\subsubsection*{Bedienung}
			Das \gls{g:myrio} benutzt als Betriebssystem ein Real Time Linux von \gls{ac:ni}. Dies bietet dem Nutzer eine Vielzahl von M�glichkeiten das \gls{g:myrio} zu bedienen \cite{lit:rt_linux}.
			%
			\begin{description}
				\item[\gls{g:labview}] Normalerweise erfolgt die Bedienung via \gls{g:labview}. Dabei werden \gls{ac:vi}s erstellt und auf das \gls{g:myrio} geladen. Zus�tzlich besteht die M�glichkeit mit Hilfe vorgefertigter \gls{ac:vi}s direkt auf Real Time Linux zuzugreifen.
				%
				\item[NI MAX]
				%
				\item[Weboberfl�che]
				%
				\item[Terminal]
			\end{description}
		%
		\paragraph*{Daten extrahieren}
	%
	\subsection{Unterscheidung Host und Target}
%
%
%DAQ in MATLAB
\section{\acrshort{ac:daq} in MATLAB}
	MATLAB bietet zum direkten Auslesen von Daten eine Data Acquisition Toolbox an. In dieser k�nnen angeschlossene \acrshort{ac:daq}-Karten in einer sogenannten Session direkt in MATLAB Beispielsweise mittels USB ausgelesen werden. Auch k�nnen Daten/Signale, welche in MATLAB generiert wurden �ber Ausg�nge der Karte ausgegeben werden.\par 
	%
	Zum Auslesen eines Beschleunigungssensors wurde ein kurzes MATLAB-Skript geschrieben (ersichtlich in \vref{ch:daqauslesen}). Dieses Skript sollte auch zuk�nftigen Arbeiten eine schnelle und einfache Verwendung der Beschleunigungssensoren erm�glichen.
%
%
%Rotationsmatrix
\section{Rotationsmatrix}
In der Linearen Algebra bietet eine Rotationsmatrix die M�glichkeit, einen Punkt im Raum um einen bestimmten Winkel $\phi$ zu drehen. In unserem Fall ist eine Rotation eines Punktes um eine Gerade im Raum n�tig. Dies wurde in der Linearen Algebra mit Herrn Jenni im zweiten Semseter behandelt \cite{lit:linalg2}.\par
Die Gerade g hat die Richtung $\overrightarrow{n}=\left(a\;b\;c\right)^{T}$ und verl�uft durch den Ursprung. Ihre Parameterdarstellung ist somit: $\overrightarrow{r}= \lambda*\overrightarrow{n}$ mit $\lambda\;\epsilon\;\mathbb{R}$. Der Punkt P mit den Koordinaten $\left(x\mid y\mid z\right)$ soll um den Winkel $\phi$ im Gegenuhrzeigersinn um die Gerade g in den Punkt P' mit Koordinaten $\left(x'\mid y'\mid z'\right)$ gedreht werden. Der Berechnungsschritt l�sst sich somit wie in \vref{eq:rot} zusammenfassen:
\formula{
            \left(\begin{array}{c}x'\\ y'\\z'\end{array}\right)=R_{g;O;\overrightarrow{n};\phi}*\left(\begin{array}{c}x\\ y\\z\end{array}\right)
        }{
            R_{g;O;\overrightarrow{n};\phi} & Rotationsmatrix um Winkel $\phi$ um die Gerade g}[eq:rot]
            
Die Rotationsmatrix wird dabei wie in \vref{eq:rotmax} aufgestellt.
\formula{
            R_{g;O;\overrightarrow{n};\phi} &= \cos(\phi)*\begin{bmatrix}1 & 0 & 0 \\0 & 1 & 0 \\ 0 & 0 & 1 \end{bmatrix}+\frac{(1-\cos(\phi))}{a^{2}+b^{2}+c^{2}}*\begin{bmatrix}a^{2} & ab & ac \\ ab & b^{2} & bc \\ ac & bc & c^{2} \end{bmatrix}\\&+\frac{\sin(\phi)}{\sqrt{a^{2}+b^{2}+c^{2}}}*\begin{bmatrix}0 & -c & b \\ c & 0 & -a \\ -b & a & 0 \end{bmatrix}
        }{ 
           }[eq:rotmax]
          
\section{nummerische Integration}
\subsection{Trapezmethode}
\subsection{Filter}