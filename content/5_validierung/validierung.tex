%%%%%%%%%%%%%%%%%%%%%%%%%%%%%%%%%%%%%%%%%%%%%%%%%%%%%%%%%%%%%%%%%%%%%%%%%%%%%%%
% Titel:   Validierung
% Autor:   zursr1, gross10
% Datum:   28.05.2014
% Version: 0.0.1
%%%%%%%%%%%%%%%%%%%%%%%%%%%%%%%%%%%%%%%%%%%%%%%%%%%%%%%%%%%%%%%%%%%%%%%%%%%%%%%
%
%:::Change-Log:::
% Versionierung erfolgt auf folgende Gegebenheiten: -1. Release Versionen
%                                                   -2. Neue Kapitel
%                                                   -3. Fehlerkorrekturen
%
% 0.0.0       Erstellung der Datei
%%%%%%%%%%%%%%%%%%%%%%%%%%%%%%%%%%%%%%%%%%%%%%%%%%%%%%%%%%%%%%%%%%%%%%%%%%%%%%%
\chapter{Validierung}\label{ch:validierung}
\section{Ergebnis der Arbeit}
Das entwickelte Konzept zeichnet die Auslenkung des Knies in einer Ebene senkrecht zur Gravitation auf. Die Kompensation der Orientierung des Sensors durch Rotation wurde erfolgreich verifiziert. Die Auslenkung des Knies in der \gls{g:transversalebene} wurde ebenfalls in einem experimentelle Umfeld ausgewertet. Dabei wurde f�r den Fall der Translation eine geeignete Performance erreicht.\par
%
Bei Auslenkungen, welche zus�tzlich eine Rotation beinhalten, erh�hen sich die Fehler durch die Projektionen der Gravitationskomponente auf die \gls{g:transversalebene}. Diese Verf�lschung wird mit integriert und f�hrt zu einem nicht akzeptablen Fehler in der Auslenkungssch�tzung. Das Konzept ist bei dieser Art von Bewegung noch fehlerhaft und es besteht diesbez�glich noch Verbesserungspotenzial.\par
%
Die theoretische Analyse der Bewegung am Knie bietet die M�glichkeit weitere Verbesserungsm�glichkeiten am Konzept aufzuzeigen. Die Bewegung kann mittels eines Modells in einen dynamischen und einen statischen Fall unterteilt werden. Das Model beschreibt die Bewegung als eine harmonische Funktion. Der dynamische Fall unterscheidet sich vom statischen durch eine definierte Grenzfrequenz. Diese ist nur abh�ngig von der Knieh�he. Im dynamischen Fall hat bei einer Auslenkung, welche eine Translation und eine Rotation beinhaltet, die projektierte Gravitation einen vernachl�ssigbar kleinen Einfluss. Dieses Modell wurde mathematisch hergeleitet und anschliessend simuliert und soll als Grundlage zur Erarbeitung zuk�nftiger Auslenkungs-Sch�tzungsmethoden dienen.\par 
%
Die Implementation wurde in \gls{g:labview} durchgef�hrt. Aufgrund der Probleme bei der Konzeptentwicklung konnte die Implementierung nicht vollst�ndig umgesetzt werden. Hier besteht das teilweise umgesetzte Konzept. Probleme bestehen momentan bei der Integration der Beschleunigungsdaten, die noch nicht zuverl�ssig funktioniert. Aus diesem Grund konnte die Implementierung auch noch nicht vollst�ndig validiert werden.
%
F�r reproduzierbare Validierungen von Konzepten hinsichtlich dieses Messumfelds sind drei verschiedene Experimental-Protokolle erstellt worden. Diese Umfassen...
%
%
% Soll/Ist-Vergleich Zeitplan
\section{Soll/Ist-Vergleich Zeitplan}\label{sec:sollist}
	Der zu Beginn im Abschnitt \vref{ss:ablauf_gliederung_projekt} geplante Ablauf konnte bezogen auf die Arbeitsteilung zu grossen Teilen eingehalten werden. Eine Ausnahme ergab sich in der Konzeptphase, bei der es wegen Verz�gerungen zwingend wurde, dass beide Autoren sich an der Entwicklung des Algorithmus und der Experimental-Protokollen beteiligten. Hinsichtlich der Beurteilbarkeit des Projekts wurde jedoch darauf geachtet, dass die Mehrheit der Arbeit demjenigen Autor zufiel, der auch die Verantwortung f�r das \acrfull{ac:ap} trug.\par
	%
	Die anget�nte Verz�gerung w�hrend der Konzeptphase hatte auch grossen Einfluss auf den definierten Zeitplan. Anstatt der vorgesehenen Aufwand von drei Tage f�r die Entwicklung des Algorithmus mussten deren 11 eingesetzt werden. Durch diesen Fakt wurden alle folgenden \gls{ac:ap}s und Meilensteine zeitlich nach hinten verschoben. Dies hatte zur Folge, dass weniger Zeit f�r diese Arbeiten aufgewendet werden konnte. Insbesonders die \gls{ac:ap} \textsf{Umsetzung \gls{g:matlab} $\rightarrow$ \gls{g:labview}}, \textsf{Soll/Ist-Vergleich} und \textsf{Zusammenfassung der Resultate} mussten stark eingeschr�nkt werden. Um die Erf�llung der Projektziele trotzdem zu erm�glichen wurden Massnahmen hinsichtlich dem entfernen der Pakete \textsf{Pr�sentation vorbereiten} und \textsf{Plakat erstellen} ergriffen\footnote{Die beiden \gls{ac:ap} m�ssen erst bei der Pr�sentation erledigt sein}. Weiter wurden die Arbeitszeiten erheblich erh�ht. Durch dieses Vorgehen konnte Zeit gutgemacht werden, was schlussendlich zum gen�genden Abschliessen aller \gls{ac:ap}s f�hrte. Die zeitliche Entwicklung des Projekts kann gut in der Abbildung \vref{img:soll_ist} erkannt werden, bei der der geplante zeitliche Verlauf blau und der tats�chliche rot hervorgehoben ist. 
	%
	\image{content/5_validierung/image/zeit_ist_soll}{scale=1}{htbp}[Zeitplan Soll- (blau)/Ist- (rot) Vergleich][Zeitplan Soll/Ist Vergleich][img:soll_ist]
	%
	%
\section{Erkenntnisse, Ausblick}\todo{Ausblick -> Wir haben einen Testaufbau erstellt}
	\begin{itemize}
		\item Das Konzept muss umgebaut werden, damit die Rotation kontinuierlich neu berechnet wird und nicht nur zu Beginn der Messung, wenn der Sensor in Ruhe ist.
		\item Die berechnete Grenzfrequenz ist f�r die Entwicklung des verfeinerten Konzeptes zu beachten. Das Konzept kann damit in eine Berechnung der Bewegung im statischen Fall sowie im dynamischen Fall unterteilt werden.
		\item In der Literaturstudie hat sich gezeigt, dass bei vielen Arbeiten zus�tzlich ein \gls{g:gyroskop} oder ein \gls{g:magnetometer} verwendet wird. Hier ist genauer abzukl�ren ob eine Verwendung von verschiedenen Sensoren sinnvoll ist.		
		\item \gls{g:labview} nicht unter Windows 8.x verwenden. Die gesamte Toolchain verh�lt sich sehr instabil, ins besonders die USB-Treiber f�r das \gls{g:myrio}. 
	\end{itemize}
