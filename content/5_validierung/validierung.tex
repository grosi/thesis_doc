%%%%%%%%%%%%%%%%%%%%%%%%%%%%%%%%%%%%%%%%%%%%%%%%%%%%%%%%%%%%%%%%%%%%%%%%%%%%%%%
% Titel:   Validierung
% Autor:   zursr1, gross10
% Datum:   28.05.2014
% Version: 0.0.1
%%%%%%%%%%%%%%%%%%%%%%%%%%%%%%%%%%%%%%%%%%%%%%%%%%%%%%%%%%%%%%%%%%%%%%%%%%%%%%%
%
%:::Change-Log:::
% Versionierung erfolgt auf folgende Gegebenheiten: -1. Release Versionen
%                                                   -2. Neue Kapitel
%                                                   -3. Fehlerkorrekturen
%
% 0.0.0       Erstellung der Datei
%%%%%%%%%%%%%%%%%%%%%%%%%%%%%%%%%%%%%%%%%%%%%%%%%%%%%%%%%%%%%%%%%%%%%%%%%%%%%%%
\chapter{Validierung}\label{ch:validierung}\todo{Ergebnis des Projektes}
\section{Ergebnis der Arbeit}
\section{Erkenntnisse}\todo{eingehen auf Literatur (Gyro, Magnetometer)}
	\begin{itemize}
		\item \gls{g:labview} nicht unter Windows 8.x verwenden. Die gesamte Toolchain verh�lt sich sehr instabil, ins besonders die USB-Treiber f�r das \gls{g:myrio}. 
	\end{itemize}
%
%
% Soll/Ist-Vergleich Zeitplan
\section{Soll/Ist-Vergleich Zeitplan}\label{sec:sollist}\todo{Bezug auf den Zeitplan, Abweichungen erkl�ren; Stand der Dinge}
	Der zu Beginn im Abschnitt \vref{ss:ablauf_gliederung_projekt} geplante Ablauf konnte bezogen auf die Arbeitsteilung zu grossen Teilen eingehalten werden. Eine Ausnahme ergab sich in der Konzeptphase, bei der es wegen Verz�gerungen zwingend wurde, dass beide Autoren sich an der Entwicklung des Algorithmus und der Experimental-Protokollen beteiligten. Hinsichtlich der Beurteilbarkeit des Projekts wurde jedoch darauf geachtet, dass die Mehrheit der Arbeit demjenigen Autor zufiel, der auch die Verantwortung f�r das \acrfull{ac:ap} trug.\par
	%
	Die anget�nte Verz�gerung w�hrend der Konzeptphase hatte auch grossen Einfluss auf den definierten Zeitplan. Anstatt der vorgesehenen Aufwand von drei Tage f�r die Entwicklung des Algorithmus mussten deren 11 eingesetzt werden. Durch diesen Fakt wurden alle folgenden \gls{ac:ap}s zeitlich nach hinten verschoben. Dies hatte zur Folge, dass weniger Zeit f�r diese Arbeiten aufgewendet werden konnte. Insbesonders die \gls{ac:ap} \textsf{Umsetzung \gls{g:matlab} $\rightarrow$ \gls{g:labview}}, \textsf{Soll/Ist-Vergleich} und \textsf{Zusammenfassung der Resultate} mussten stark eingeschr�nkt werden. Um die Erf�llung der Projektziele trotzdem zu erm�glichen wurden Massnahmen hinsichtlich dem entfernen der Pakete \textsf{Pr�sentation vorbereiten} und \textsf{Plakat erstellen} ergriffen\footnote{Die beiden \gls{ac:ap} m�ssen erst bei der Pr�sentation erledigt sein}. Weiter wurden die Arbeitszeiten erheblich erh�ht. Durch dieses Vorgehen konnte Zeit gutgemacht werden, was schlussendlich zum gen�genden Abschliessen aller \gls{ac:ap}s f�hrte. Die zeitliche Entwicklung des Projekts kann gut in der Abbildung \vref{img:soll_ist} erkannt werden, bei der der geplante zeitliche Verlauf blau und der tats�chliche rot hervorgehoben ist. 
	%
	\image{content/5_validierung/image/zeit_ist_soll}{scale=1}{htbp}[Zeitplan Soll- (blau)/Ist- (rot) Vergleich][Zeitplan Soll/Ist Vergleich][img:soll_ist]
	%
	
	
\section{Ausblick}