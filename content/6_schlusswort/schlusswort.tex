%%%%%%%%%%%%%%%%%%%%%%%%%%%%%%%%%%%%%%%%%%%%%%%%%%%%%%%%%%%%%%%%%%%%%%%%%%%%%%%
% Titel:   Schlusswort
% Autor:   zursr1, gross10
% Datum:   28.05.2014
% Version: 0.0.1
%%%%%%%%%%%%%%%%%%%%%%%%%%%%%%%%%%%%%%%%%%%%%%%%%%%%%%%%%%%%%%%%%%%%%%%%%%%%%%%
%
%:::Change-Log:::
% Versionierung erfolgt auf folgende Gegebenheiten: -1. Release Versionen
%                                                   -2. Neue Kapitel
%                                                   -3. Fehlerkorrekturen
%
% 0.0.0       Erstellung der Datei
%%%%%%%%%%%%%%%%%%%%%%%%%%%%%%%%%%%%%%%%%%%%%%%%%%%%%%%%%%%%%%%%%%%%%%%%%%%%%%%    
\chapter{Schlusswort}\label{ch:schlusswort}
	Unter dem Kapitel Schlussfolgerung analysieren Sie die gewonnenen Ergebnisse und vergleichen diese eventuell mit Bekanntem.\par
	%
	Erreichtes und/oder Nicht-Erreichtes werden herausgearbeitet und mit der urspr�nglichen Zielsetzung verglichen.\par 
	%
	Die Schlussfolgerung stellt nochmals einen �berblick des Hauptteils dar. Sie fassen die in Einzelschritten gewonnenen Ergebnisse zusammen, bewerten ihren Nutzen und bestimmen ihren Stellenwert f�r die allgemeine Forschungslage.