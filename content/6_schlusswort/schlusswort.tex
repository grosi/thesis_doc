%%%%%%%%%%%%%%%%%%%%%%%%%%%%%%%%%%%%%%%%%%%%%%%%%%%%%%%%%%%%%%%%%%%%%%%%%%%%%%%
% Titel:   Schlusswort
% Autor:   zursr1, gross10
% Datum:   28.05.2014
% Version: 0.0.1
%%%%%%%%%%%%%%%%%%%%%%%%%%%%%%%%%%%%%%%%%%%%%%%%%%%%%%%%%%%%%%%%%%%%%%%%%%%%%%%
%
%:::Change-Log:::
% Versionierung erfolgt auf folgende Gegebenheiten: -1. Release Versionen
%                                                   -2. Neue Kapitel
%                                                   -3. Fehlerkorrekturen
%
% 0.0.0       Erstellung der Datei
%%%%%%%%%%%%%%%%%%%%%%%%%%%%%%%%%%%%%%%%%%%%%%%%%%%%%%%%%%%%%%%%%%%%%%%%%%%%%%%    
\chapter{Schlusswort}\label{ch:schlusswort}
	Bedingt durch das Engagement im Eurobot-Projekt standen uns f�r diese Arbeit lediglich f�nf Wochen zur Verf�gung. In dieser Zeit hatten wir uns das n�tige Wissen anzueignen, die Aufgabenstellung zu analysieren und die gestellten Anforderungen umzusetzen. Wie es sich w�hrend der Projektphasen zeigte, reichte diese Zeit nicht aus, um uns in die relevanten Aspekte der Biomechanik geb�hrend einzulesen und gleichzeitig eine Umsetzung anzustreben. Deshalb mussten wir bei der Algorithmus-Entwicklung wiederholt �nderungen anbringen, was schliesslich eine solide Realisierung verz�gerte. \\
	Und dennoch: Das ganze Projekt war eine spannende und durchaus interessante Herausforderung. Die Parallelit�t von Eurobot-Projekt und Thesis versetzte uns mental in die reale Arbeitswelt, wo h�ufig auch mehrere Projekte mit erster Priorit�t zu behandeln sind. Die zeitliche Koordination von Eurobot-Wettkampf und Bachelor-Thesis haben wir suboptimal gel�st.\par
	%
	Abschliessend l�sst sich sagen, dass in der kurzen Zeit dennoch fundierte Grundlagen f�r den weiteren Verlauf des Projekts erarbeitet und aufgezeigt werden konnten. Unsere Arbeit ist eine valable Grundlage f�r eine allf�llige Master-Arbeit.
	
	