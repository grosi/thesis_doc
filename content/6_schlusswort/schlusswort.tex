%%%%%%%%%%%%%%%%%%%%%%%%%%%%%%%%%%%%%%%%%%%%%%%%%%%%%%%%%%%%%%%%%%%%%%%%%%%%%%%
% Titel:   Schlusswort
% Autor:   zursr1, gross10
% Datum:   28.05.2014
% Version: 0.0.1
%%%%%%%%%%%%%%%%%%%%%%%%%%%%%%%%%%%%%%%%%%%%%%%%%%%%%%%%%%%%%%%%%%%%%%%%%%%%%%%
%
%:::Change-Log:::
% Versionierung erfolgt auf folgende Gegebenheiten: -1. Release Versionen
%                                                   -2. Neue Kapitel
%                                                   -3. Fehlerkorrekturen
%
% 0.0.0       Erstellung der Datei
%%%%%%%%%%%%%%%%%%%%%%%%%%%%%%%%%%%%%%%%%%%%%%%%%%%%%%%%%%%%%%%%%%%%%%%%%%%%%%%    
\chapter{Schlusswort}\label{ch:schlusswort}
	Bedingt durch das Engagement im Projekt Eurobot standen f�r diese Arbeit nur 5 Wochen zur Verf�gung. In dieser Zeit galt es sich das n�tige Wissen anzueignen, zu analysieren und die gestellten Anforderungen umzusetzen. Wie es sich w�hrend den Projektphasen zeigte, reicht diese Zeit bei weitem nicht aus um die relevanten Aspekte der Biomechanik zu verstehen und gleichzeitig eine Umsetzung anzustreben. Durch dies bedingt musste gerade bei der Algorithmus-Entwicklung wiederholt von vorne begonnen werden, was schlussendlich eine solide Realisierung verunm�glichte. Weiter w�ren klare und im Voraus bekannte Ziele und Anforderungen w�nschenswert gewesen, h�tte es doch den Einstieg in die Thematik stark vereinfacht Es ist zu hoffen, dass in naher Zukunft das Projekt Eurobot besser mit der folgenden Thesis-Arbeiten harmoniert.\par
	%
	Abschliessend l�sst sich sagen, dass in der kurzen Zeit dennoch fundierte Grundlagen f�r den weiteren Verlauf des Projekts erarbeitet und aufgezeigt werden konnten.
	
	