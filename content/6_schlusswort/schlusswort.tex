%%%%%%%%%%%%%%%%%%%%%%%%%%%%%%%%%%%%%%%%%%%%%%%%%%%%%%%%%%%%%%%%%%%%%%%%%%%%%%%
% Titel:   Schlusswort
% Autor:   zursr1, gross10
% Datum:   28.05.2014
% Version: 0.0.1
%%%%%%%%%%%%%%%%%%%%%%%%%%%%%%%%%%%%%%%%%%%%%%%%%%%%%%%%%%%%%%%%%%%%%%%%%%%%%%%
%
%:::Change-Log:::
% Versionierung erfolgt auf folgende Gegebenheiten: -1. Release Versionen
%                                                   -2. Neue Kapitel
%                                                   -3. Fehlerkorrekturen
%
% 0.0.0       Erstellung der Datei
%%%%%%%%%%%%%%%%%%%%%%%%%%%%%%%%%%%%%%%%%%%%%%%%%%%%%%%%%%%%%%%%%%%%%%%%%%%%%%%    
\chapter{Schlusswort}\label{ch:schlusswort}
	Bedingt durch das Engagement im Eurobot-Projekt standen uns f�r diese Arbeit nur 5 Wochen zur Verf�gung. In dieser Zeit hatten wir uns das n�tige Wissen anzueignen, die Aufgabenstellung zu analysieren und die gestellten Anforderungen umzusetzen. Wie es sich w�hrend den Projektphasen zeigte, reichte diese Zeit nicht aus, um uns in die relevanten Aspekte der Biomechanik geb�hrend einzulesen und gleichzeitig eine Umsetzung anzustreben. Durch dies bedingt musste wir bei der Algorithmus-Entwicklung wiederholt Erneuerungen anbringen, was schlussendlich eine solide Realisierung verunm�glichte. Weiter w�ren klare und im Voraus bekannte Ziele und Anforderungen w�nschenswert gewesen. Wir verstehen, dass dies bei einem Forschungsprojekt nicht m�glich ist. Dies h�tte uns jedoch den Einstieg in die Thematik stark vereinfacht.\par
%	
Es ist weiter zu hoffen, dass in naher Zukunft das Eurobot-Projekt besser mit der folgenden Thesis-Arbeiten harmoniert. Dies w�rde eine gr�ssere Zufriedenstellung f�r die Studierenden sowie die betreuenden Dozenten sicherstellen.\par
	%
	Abschliessend l�sst sich sagen, dass in der kurzen Zeit dennoch fundierte Grundlagen f�r den weiteren Verlauf des Projekts erarbeitet und aufgezeigt werden konnten.
	
	