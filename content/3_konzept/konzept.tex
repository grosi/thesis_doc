%%%%%%%%%%%%%%%%%%%%%%%%%%%%%%%%%%%%%%%%%%%%%%%%%%%%%%%%%%%%%%%%%%%%%%%%%%%%%%%
% Titel:   Konzept
% Autor:   zursr1, gross10
% Datum:   28.05.2014
% Version: 0.0.1
%%%%%%%%%%%%%%%%%%%%%%%%%%%%%%%%%%%%%%%%%%%%%%%%%%%%%%%%%%%%%%%%%%%%%%%%%%%%%%%
%
%:::Change-Log:::
% Versionierung erfolgt auf folgende Gegebenheiten: -1. Release Versionen
%                                                   -2. Neue Kapitel
%                                                   -3. Fehlerkorrekturen
%
% 0.0.0       Erstellung der Datei
%%%%%%%%%%%%%%%%%%%%%%%%%%%%%%%%%%%%%%%%%%%%%%%%%%%%%%%%%%%%%%%%%%%%%%%%%%%%%%%
\chapter{Konzept}\label{ch:konzept}
%
%
%Datenaufnahme mit ICP-Beschleunigungssensoren
\section{Datenaufnahme mit ICP-Beschleunigungssensoren}
Um bei der Konzeptentwicklung Daten zum �berpr�fen des Algorithmus zu haben, haben wir von Rolf Vetter \gls{ac:icp}-Beschleunigungssensoren von IMI Sensors erhalten. Mit einer \gls{ac:daq}-Karte von \gls{ac:ni}, der \gls{ac:ni} USB-4431, wurden die Beschleunigungssensoren direkt mittels MATLAB ausgelesen. Somit sind die Daten direkt in MATLAB zur Weiterverarbeitung abgelegt. Es hat sich jedoch schnell herausgestellt, dass die Sensoren f�r unsere Anwendung nicht geeignet sind. Der Frequenzbereich des Sensors ist im Datenblatt in \vref{ch:indusriesensor} ersichtlich. Dort zu sehen, dass der Sensor einen Frequenzbereich von 0.5 Hz bis 10 kHz hat. Somit wird die f�r uns notwendige Gravitationsbeschleunigung, die einem DC-Anteil im Signal entspricht, nicht aufgenommen. Diese Sensoren k�nnen somit nur f�r dynamische Anwendungen verwendet werden.
%
%
%Algorithmus
\section{Algorithmus}\label{s:konzept_algo}
	%
	\subsection{�bersicht}
	%
	\subsection{Filterung}
	%
	\subsection{Winkeldetektion}
	%
	\subsection{Rotation}
	%
	\subsection{Integration}
%
%
%Software
\section{Software}\label{s:konzept:sw}
	%
	\subsection{Rohdaten aufzeichnen}
	%
	\subsection{myRIO Posture Estimation}
%
%
%Experimental-Protokoll
\section{Experimental-Protokoll}\label{s:experimental_protokoll}
	%
	%
	\subsection{Grund/Zweck der Messung}
	%
	%
	\subsection{Material/Testumgebung}
	%
	%
	\subsection{Methoden}
		\subsubsection*{Messgruppen}\todo{unabh�ngige Variabel}
		\subsubsection*{Testbeschreibungen}\todo{Ablauf}
		\subsubsection*{Datenauswertung}\todo{abh�ngige Variabel}
	%
	%
	\subsection{Kontrollgruppe}
	%
	%
	\subsection{Interpretation}\todo{Histogramm, Gruppenvergleich,Messfehler}
%
%
%Validierung
\section{Validierung}
\todo{Auswertung Konzept}
