%%%%%%%%%%%%%%%%%%%%%%%%%%%%%%%%%%%%%%%%%%%%%%%%%%%%%%%%%%%%%%%%%%%%%%%%%%%%%%%
% Titel:   Konzept
% Autor:   zursr1, gross10
% Datum:   28.05.2014
% Version: 0.0.1
%%%%%%%%%%%%%%%%%%%%%%%%%%%%%%%%%%%%%%%%%%%%%%%%%%%%%%%%%%%%%%%%%%%%%%%%%%%%%%%
%
%:::Change-Log:::
% Versionierung erfolgt auf folgende Gegebenheiten: -1. Release Versionen
%                                                   -2. Neue Kapitel
%                                                   -3. Fehlerkorrekturen
%
% 0.0.0       Erstellung der Datei
%%%%%%%%%%%%%%%%%%%%%%%%%%%%%%%%%%%%%%%%%%%%%%%%%%%%%%%%%%%%%%%%%%%%%%%%%%%%%%%
\chapter{Konzept}\label{ch:konzept}
%
% Blockdiagramm
\section{Blockdiagramm}
%
%Algorithmus
\section{Algorithmus}\label{s:konzept_algo}
	%
	\subsection{�bersicht}
	%
	\subsection{Filterung}
	%
	\subsection{Winkeldetektion}
	%
	\subsection{Rotation}
	%
	\subsection{Integration}
%
%
%Software
\section{Software}\label{s:konzept:sw}
	%
	\subsection{Rohdaten aufzeichnen}\label{ss:rohdaten_aufzeichnen}
	%
	\subsection{myRIO Posture Estimation}
%
%
%Verifikation Translation
\section{Verifikation Translation}\label{s:verifikation_translation}
	%
	%
	\subsection{Grund/Zweck der Messung}
		Die Verifikation der Translation stellt das erste von drei Experimental-Protokollen\footnote{die �brigen Protokolle sind in den Abschnitten \vref{s:verifikation_rotation} und \vref{s:verifikation_transaltion_rotation} zu finden} dar, die folgend f�r die Validierung des Konzepts als Ausgangslagen dienen sollen.  Ziel der Messung ist es den Algorithmus hinsichtlich einer linearen Translation zu �berpr�fen. Dabei gilt es relevante Fakten wie Genauigkeit und Dynamik zu extrahieren und f�r die Validierung festzuhalten. Bei auftretenden Diskrepanzen muss das Konzept �berarbeitet und dieses Protokoll wiederholt werden. Das Erf�llen	
	%
	%
	\subsection{Material/Testumgebung}
		\begin{table}[htbp]
     \centering
     \caption{Messmittelliste Verifikation Translation}
     \label{tab:messmittel_translation}
     \begin{tabularx}{\textwidth}{|l|X|l|} 
         \hline
         \rowcolor{bfhblue}
         \textcolor{white}{Messmittel} & \textcolor{white}{Beschreibung} & \textcolor{white}{gesch�tzte Messgenauigkeit} \\
         \hline
         M\low{1} & \gls{g:myrio} 1900, Serial Number 03036833 & \unit[3.91]{mg} (\unit[1]{g} = \unitfrac[9.81]{m}{s\high{2}})\\
         \hline
         M\low{2} & KDS Massstab \unit[50]{cm}, 60.3.12 & \unit[200]{$\mu$m}\\
         \hline
         M\low{3} & Schaublin-Velleneuve, WM0538-1 & \unit[20]{$\mu$m}\\
         \hline
         M\low{4} & Stoppuhr App, HTC One silver & \unit[10]{ms}\\
         \hline
     \end{tabularx}  
\end{table}	
		\image{content/3_konzept/image/messaufbau}{scale=0.6}{htbp}[Messaufbau - Dimensionen][img:messaufbau_dimensionen]
		\begin{table}[htbp]
     \centering
     \caption{Messaufbau - Gr�ssen}
     \label{tab:messaufbau_groessen}
     \begin{tabularx}{\textwidth}{|l|l|X|} 
         \hline
         \rowcolor{bfhblue}
         \textcolor{white}{Bezeichnung} & \textcolor{white}{Bereich} & \textcolor{white}{Bemerkung} \\
         \hline
         x & \unit[47]{mm} & gemessen mit einem Lineal\\
         \hline
         y & \unit[515]{mm} & gemessen mit einem Lineal\\
         \hline
         r & \unit[517.14]{mm} & berechnet nach Gleichung \vref{eq:radius}\\
         \hline
         b & \unit[189.54]{mm} & berechnet nach Gleichung \vref{eq:kreissegement} \\
         \hline
         $\alpha$ & \unit[21]{�} & gemessen mit M\low{1} \\
         \hline
     \end{tabularx}  
\end{table}
	%
	%
	\subsection{Methoden}
		%
		\subsubsection*{Messgruppen}\todo{unabh�ngige Variabel}
			Es werden zwei Messgruppen definiert. Die eine dient als Testgruppe und die andere als Kontrollgruppe. 
			%
			\begin{itemize}
				%
				\item \textbf{Testgruppe} Die Testgruppe repr�sentiert das \gls{g:myrio}, das zur Aufzeichnung der Beschleunigungsdaten ben�tigt wird. Dabei werden die Daten nur gespeichert und nicht verarbeitet\footnote{F�r detailliertere Informationen zur Aufzeichnung sei auf den Abschnitt \vref{ss:rohdaten_aufzeichnen} verwiesen}. Die Signalverarbeitung erfolgt zu einem sp�teren Zeitpunkt auf einem Computer mit Hilfe von \gls{g:matlab}.
				%
				\item \textbf{Kontrollgruppe} Die zweite Gruppe stellt die direkte Messung mit Massstab dar. Sie ist gleichzeitig die Kontrollgruppe, die schlussendlich bei der Interpretation als Ausgangslage gilt.
			\end{itemize}
			%
			Die Messmethodik stellt bei den beschriebenen Gruppen die unabh�ngige Variabel dar.
		%
		\subsubsection*{Testbeschreibungen}\todo{Ablauf}
		\subsubsection*{Datenauswertung}\todo{abh�ngige Variabel}
		\formula{
			r &= \sqrt{x^2+y^2}\\
			&=\sqrt{47^2+515^2}=\underline{517.14}
		}
		{
			r & Radius in mm\\
			x & vertikaler Abstand in mm\\
			y & horizontaler Abstand in mm\\
		}[eq:radius]
		%
		\formula{
			\alpha ' &= \frac{\alpha}{180}\cdot\pi\\[1ex]
			&= \frac{21}{180}\cdot\pi = \underline{0.366}\\[2ex]
			b&=r\cdot\alpha '\\
			&=517.14\cdot 0.366=\underline{189.54}
		}
		{
			r & Radius in mm\\
			\alpha & Winkel in Grad\\
			\alpha ' & Winkel in Rad\\
			b & Teilumfang in mm\\
		}[eq:kreissegement]
	%
	%
	\subsection{Kontrollgruppe}
	%
	%
	\subsection{Interpretation}\todo{Histogramm, Gruppenvergleich,Messfehler}
%
%
%Verifikation Rotation
\section{Verifikation Rotation}\label{s:verifikation_rotation}
	%
	%
	\subsection{Grund/Zweck der Messung}
	%
	%
	\subsection{Material/Testumgebung}
		Zur Aufzeichnung eines Bewegungsablaufes des Knies wurde eine Mechanik zur Simulation erstellt. Sie basiert auf Bosch-Profilen\footnote{Hergestellt von der Firma Bosch Rexroth AG: \url{http://www.boschrexroth.com}} und ist durch das sehr anpassbar. Die Simulation erm�glicht es einen statischen Bewegungsablauf nachzubilden. Dabei wird aktuell nur die seitliche Auslenkung des Knies ber�cksichtigt. Die ebenfalls vorhandene Rotation wird in diesem Projekt nicht
		\begin{table}[htbp]
     \centering
     \caption{Messmittelliste Verifikation Translation}
     \label{tab:messmittel_translation}
     \begin{tabularx}{\textwidth}{|l|X|l|} 
         \hline
         \rowcolor{bfhblue}
         \textcolor{white}{Messmittel} & \textcolor{white}{Beschreibung} & \textcolor{white}{gesch�tzte Messgenauigkeit} \\
         \hline
         M\low{1} & \gls{g:myrio} 1900, Serial Number 03036833 & \unit[3.91]{mg} (\unit[1]{g} = \unitfrac[9.81]{m}{s\high{2}})\\
         \hline
         M\low{2} & KDS Massstab \unit[50]{cm}, 60.3.12 & \unit[200]{$\mu$m}\\
         \hline
         M\low{3} & Schaublin-Velleneuve, WM0538-1 & \unit[20]{$\mu$m}\\
         \hline
         M\low{4} & Stoppuhr App, HTC One silver & \unit[10]{ms}\\
         \hline
     \end{tabularx}  
\end{table}	
		\image{content/3_konzept/image/messaufbau}{scale=0.6}{htbp}[Messaufbau - Dimensionen][img:messaufbau_dimensionen]
		\begin{table}[htbp]
     \centering
     \caption{Messaufbau - Gr�ssen}
     \label{tab:messaufbau_groessen}
     \begin{tabularx}{\textwidth}{|l|l|X|} 
         \hline
         \rowcolor{bfhblue}
         \textcolor{white}{Bezeichnung} & \textcolor{white}{Bereich} & \textcolor{white}{Bemerkung} \\
         \hline
         x & \unit[47]{mm} & gemessen mit einem Lineal\\
         \hline
         y & \unit[515]{mm} & gemessen mit einem Lineal\\
         \hline
         r & \unit[517.14]{mm} & berechnet nach Gleichung \vref{eq:radius}\\
         \hline
         b & \unit[189.54]{mm} & berechnet nach Gleichung \vref{eq:kreissegement} \\
         \hline
         $\alpha$ & \unit[21]{�} & gemessen mit M\low{1} \\
         \hline
     \end{tabularx}  
\end{table}
	%
	%
	\subsection{Methoden}
		%
		\subsubsection*{Messgruppen}\todo{unabh�ngige Variabel}
			
		%
		\subsubsection*{Testbeschreibungen}\todo{Ablauf}
		\subsubsection*{Datenauswertung}\todo{abh�ngige Variabel}
		\formula{
			r &= \sqrt{x^2+y^2}\\
			&=\sqrt{47^2+515^2}=\underline{517.14}
		}
		{
			r & Radius in mm\\
			x & vertikaler Abstand in mm\\
			y & horizontaler Abstand in mm\\
		}[eq:radius]
		%
		\formula{
			\alpha ' &= \frac{\alpha}{180}\cdot\pi\\[1ex]
			&= \frac{21}{180}\cdot\pi = \underline{0.366}\\[2ex]
			b&=r\cdot\alpha '\\
			&=517.14\cdot 0.366=\underline{189.54}
		}
		{
			r & Radius in mm\\
			\alpha & Winkel in Grad\\
			\alpha ' & Winkel in Rad\\
			b & Teilumfang in mm\\
		}[eq:kreissegement]
	%
	%
	\subsection{Kontrollgruppe}
	%
	%
	\subsection{Interpretation}\todo{Histogramm, Gruppenvergleich,Messfehler}
%
%
%Verifikation Translation mit anschliessender Rotation
\section{Verifikation Translation mit anschliessender Rotation}\label{s:verifikation_transaltion_rotation}
	%
	%
	\subsection{Grund/Zweck der Messung}
	%
	%
	\subsection{Material/Testumgebung}
		Zur Aufzeichnung eines Bewegungsablaufes des Knies wurde eine Mechanik zur Simulation erstellt. Sie basiert auf Bosch-Profilen\footnote{Hergestellt von der Firma Bosch Rexroth AG: \url{http://www.boschrexroth.com}} und ist durch das sehr anpassbar. Die Simulation erm�glicht es einen statischen Bewegungsablauf nachzubilden. Dabei wird aktuell nur die seitliche Auslenkung des Knies ber�cksichtigt. Die ebenfalls vorhandene Rotation wird in diesem Projekt nicht
		\begin{table}[htbp]
     \centering
     \caption{Messmittelliste Verifikation Translation}
     \label{tab:messmittel_translation}
     \begin{tabularx}{\textwidth}{|l|X|l|} 
         \hline
         \rowcolor{bfhblue}
         \textcolor{white}{Messmittel} & \textcolor{white}{Beschreibung} & \textcolor{white}{gesch�tzte Messgenauigkeit} \\
         \hline
         M\low{1} & \gls{g:myrio} 1900, Serial Number 03036833 & \unit[3.91]{mg} (\unit[1]{g} = \unitfrac[9.81]{m}{s\high{2}})\\
         \hline
         M\low{2} & KDS Massstab \unit[50]{cm}, 60.3.12 & \unit[200]{$\mu$m}\\
         \hline
         M\low{3} & Schaublin-Velleneuve, WM0538-1 & \unit[20]{$\mu$m}\\
         \hline
         M\low{4} & Stoppuhr App, HTC One silver & \unit[10]{ms}\\
         \hline
     \end{tabularx}  
\end{table}	
		\image{content/3_konzept/image/messaufbau}{scale=0.6}{htbp}[Messaufbau - Dimensionen][img:messaufbau_dimensionen]
		\begin{table}[htbp]
     \centering
     \caption{Messaufbau - Gr�ssen}
     \label{tab:messaufbau_groessen}
     \begin{tabularx}{\textwidth}{|l|l|X|} 
         \hline
         \rowcolor{bfhblue}
         \textcolor{white}{Bezeichnung} & \textcolor{white}{Bereich} & \textcolor{white}{Bemerkung} \\
         \hline
         x & \unit[47]{mm} & gemessen mit einem Lineal\\
         \hline
         y & \unit[515]{mm} & gemessen mit einem Lineal\\
         \hline
         r & \unit[517.14]{mm} & berechnet nach Gleichung \vref{eq:radius}\\
         \hline
         b & \unit[189.54]{mm} & berechnet nach Gleichung \vref{eq:kreissegement} \\
         \hline
         $\alpha$ & \unit[21]{�} & gemessen mit M\low{1} \\
         \hline
     \end{tabularx}  
\end{table}
	%
	%
	\subsection{Methoden}
		%
		\subsubsection*{Messgruppen}\todo{unabh�ngige Variabel}
			
		%
		\subsubsection*{Testbeschreibungen}\todo{Ablauf}
		\subsubsection*{Datenauswertung}\todo{abh�ngige Variabel}
		\formula{
			r &= \sqrt{x^2+y^2}\\
			&=\sqrt{47^2+515^2}=\underline{517.14}
		}
		{
			r & Radius in mm\\
			x & vertikaler Abstand in mm\\
			y & horizontaler Abstand in mm\\
		}[eq:radius]
		%
		\formula{
			\alpha ' &= \frac{\alpha}{180}\cdot\pi\\[1ex]
			&= \frac{21}{180}\cdot\pi = \underline{0.366}\\[2ex]
			b&=r\cdot\alpha '\\
			&=517.14\cdot 0.366=\underline{189.54}
		}
		{
			r & Radius in mm\\
			\alpha & Winkel in Grad\\
			\alpha ' & Winkel in Rad\\
			b & Teilumfang in mm\\
		}[eq:kreissegement]
	%
	%
	\subsection{Kontrollgruppe}
	%
	%
	\subsection{Interpretation}\todo{Histogramm, Gruppenvergleich,Messfehler}
%
%
%Validierung
\section{Validierung}
\todo{Auswertung Konzept}
