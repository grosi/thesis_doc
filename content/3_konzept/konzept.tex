%%%%%%%%%%%%%%%%%%%%%%%%%%%%%%%%%%%%%%%%%%%%%%%%%%%%%%%%%%%%%%%%%%%%%%%%%%%%%%%
% Titel:   Konzept
% Autor:   zursr1, gross10
% Datum:   28.05.2014
% Version: 0.0.1
%%%%%%%%%%%%%%%%%%%%%%%%%%%%%%%%%%%%%%%%%%%%%%%%%%%%%%%%%%%%%%%%%%%%%%%%%%%%%%%
%
%:::Change-Log:::
% Versionierung erfolgt auf folgende Gegebenheiten: -1. Release Versionen
%                                                   -2. Neue Kapitel
%                                                   -3. Fehlerkorrekturen
%
% 0.0.0       Erstellung der Datei
%%%%%%%%%%%%%%%%%%%%%%%%%%%%%%%%%%%%%%%%%%%%%%%%%%%%%%%%%%%%%%%%%%%%%%%%%%%%%%%
\chapter{Konzept}\label{ch:konzept}
\section{Datenaufnahme mit \acrshort{ac:icp}-Beschleunigungssensoren}
Um bei der Konzeptentwicklung Daten zum �berpr�fen des Algorithmus zu haben, haben wir von Rolf Vetter \gls{ac:icp}-Beschleunigungssensoren von IMI Sensors erhalten. Mit einer \gls{ac:daq}-Karte von \gls{ac:ni}, der \gls{ac:ni} USB-4431, wurden die Beschleunigungssensoren direkt mittels MATLAB ausgelesen. Somit sind die Daten direkt in MATLAB zur Weiterverarbeitung abgelegt. Es hat sich jedoch schnell herausgestellt, dass die Sensoren f�r unsere Anwendung nicht geeignet sind. Der Frequenzbereich des Sensors ist im Datenblatt in \vref{ch:indusriesensor} ersichtlich. Dort zu sehen, dass der Sensor einen Frequenzbereich von 0.5 Hz bis 10 kHz hat. Somit wird die f�r uns notwendige Gravitationsbeschleunigung, die einem DC-Anteil im Signal entspricht, nicht aufgenommen. Diese Sensoren k�nnen somit nur f�r dynamische Anwendungen verwendet werden.
\section{Algorithmus}
\subsection{�bersicht}
\subsection{Filterung}
\subsection{Winkeldetektion}
\subsection{Rotation}
\subsection{Integration}
\section{Software}
\section{Experimental-Protokoll }
\subsection{Mechanik}
\subsection{Messhardware}
\subsection{Software}
\subsection{Messfehler}
\subsection{Ablauf}
\todo{verschiedene Messung beschieben f�r Reprodutzierbarkeit}
\section{Validierung}
\todo{Auswertung Experimental-Protokoll}
